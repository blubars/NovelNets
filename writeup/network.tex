\section{Why a Network?}

There are many possible definitions of an edge in a social network: dialogue, physical interactions, sentiment. 
We chose to present character co-mentions in the text as edges. 
This gives us the broadest view of a book as a network of character interactions, and additionally eases the burden of natural-language parsing because we can disregard all semantic information in the interactions.

\subsection{Questions..questions that need answering}

How is the narrative world structured? (static-network analysis)
Main characters: centrality measures
Small-world, community detection, assortativity


How does the structure evolve throughout the narrative? (dynamic analysis)
Chronological order vs. book order
Attachment: sparsification/densification?
Main character centralities over time


Why is the book structured this way?
Chronological vs. non-chronological
How does the structure impact the narrative?


\subsection{Related Work}

A quick search shows that we are not the first to mine networks from literary works and apply network analysis techniques.

In fact, there are a few interesting works which use data-driven analysis to evaluate hypotheses in literary analysis. 
Though it doesn't use networks, Reagan et al.'s uses NLP and sentiment analysis to confirm Kurt Vonnegut's rejected Master's thesus on emotional story arcs \cite{Reagan2016}. 
Analyzing 1,327 stories from Project Gutenberg, they find six core emotional trajectory arcs and compare modern popularity of stories for each arc evaluated by number of downloads.

Several social networks have been built from narratives. 
Alberich et al. (2002) build bipartite social network out of Marvel comic book character appearances in comic books, finding many similar characteristics to real-life scientific collaboration networks such as short distances and a power-law degree distribution tail with a cutoff \cite{2002marvel}. Notably, they find a lower clustering coefficient than expected in real networks.

Bonato et al. \cite{Bonato2016} compared networks for three books: {\em Twilight} by Stephanie Meyer, {\em The Stand} by Steven King, and J.K. Rowling's {\em Harry Potter and the Goblet of Fire}. 
They found the Chung-Lu model best fits the co-occurance networks.
Game of Thrones and Lord of the Rings have also been similarly analyzed \cite{GOT,ribeiro2016complex}.

%Meanwhile, Sack (1998) examines the emergence of global narrative structures from the local interactions of stable and unstable character triads, and discusses the implications as a generative mechanism for new story creation, particularly for dramas \cite{sack2013character}. Befriending and betrayal interactions between nodes represent an interesting potential dynamic for understanding narrative structure.

\infinitejest differs from the other narratives touched on in prior works because of its complexity, supposedly wide and in-depth wordview, and its unusual (non-chronological) structure. We therefore hope to find how this artificial social network compares to real social networks and to those previously analyzed. We suspect we may see statistics more similar to real-world social networks.

