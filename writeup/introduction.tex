\section{Introduction}

Does a story mirror the world, or show a different one? How does a story's structure impact its narrative?

We pursue an analysis of David Foster Wallace's \infinitejest to answer variants of these two questions, viewing the book as a growing network:

\begin{enumerate}
    \item Does the story's social network mirror the world or show a different one?
    \item Why did the author choose to structure the narrative out of chronological order?
\end{enumerate}

\subsection{A mirror of the world?}

\infinitejest is of sufficient length to present a complex social network of characters and interactions between them. We make qualitative comparisons of its network of characters and their interactions with those one could expect to find in real-world social networks and random graphs. 

Mediums for storytelling have evolved from an oral tradition to a range of genres with the written word and invention of the printing press. 

\subsection{How does structure impact narrative?}

It's a non-statement to say that a novel's structure impacts its narrative, but to answer ``how?'' is very difficult. \infinitejest, in particular, reveals the narrative world in pieces and leaves it as an exercise for the reader to stitch together the complete picture and maintain an updated view as the story progresses. 

With this in mind, we may additionally examine the relationship between events that expose the story's world (expositions) building and driving the plot forward. 



Does this choice have a purpose reflected in the network structure which might make the book more or less comprehensible to the reader, or is it purely stylistic?

TODO: cite papers on narrative structures? 

In this work we seek to expand network analysis techniques for real-world social networks to examine artificial social networks in narratives.

Specifically, we extract and analyze a network of character interactions from \infinitejest, using character co-occurances to produce a growing weighted undirected graph where higher edge weights indicating more interactions.

We compare this narrative social network to properties exhibited by real-world networks, including a heavy-tailed degree distribution, the small-world property, inverse degree assortativity, and community structure.

We find that centrality measures such as degree centrality and betweenness accurately identify the main characters, and comment on gender bias in the network.

Finally, we examine the narrative structure of \infinitejest by comparing the dynamics of the growing network in authored book-order versus chronological order.

