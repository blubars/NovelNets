\section{Introduction}

Does a story mirror the world, or show a different one? How does a story's structure impact its narrative?

We analyze David Foster Wallace's \infinitejest as a network in an attempt to answer variants of these two questions:

\begin{enumerate}
    \item Does the story's social network mirror the world or show a different one?
    \item Why did the author choose to structure the narrative out of chronological order?
\end{enumerate}

To answer these questions, we extract a network of character interactions that grows as the book progresses. We pursue our analyses in two contexts:

\begin{enumerate}
    \item \textbf{Statically}: analyzing the state of the book's network at the time of its end. 
    \item \textbf{Dynamically}: analyzing the state of the book's network as it grows throughout the telling.
\end{enumerate}

\subsection{A mirror of the world?}

There is a complex social network of characters and their interactions in \infinitejest. We make qualitative comparisons of its network of characters and their interactions with those one could expect to find in real-world social networks and random graphs. 

We compare this narrative social network to properties exhibited by real-world networks, examining its degree distribution, inverse degree assortativity, and community structure.

We find that centrality measures -- degree centrality and betweenness specifically -- accurately identify important characters. 

We also explore gender bias in the network, an aspect of the book that has received significant criticism.

\subsection{How does structure impact narrative?}

In \infinitejest the world is revealed only in pieces and indirectly, and it is up to the reader to stitch those pieces together to create complete picture and to maintain a coherent view of the story as it progresses. 

It's a non-statement to say that a novel's structure impacts its narrative, but it is far from so easy to answer the question ``how?'' In an effort to answer this question quantitatively, we ask: ``does the decision to tell the story out of chronological order make it more or less comprehensible for the reader?''

Our examination of the book's chronology leads us to a new understanding of the kinds of functions events in the book take on and encourages a distinction between events that expose the world (functioning as exposition), and those that drive the narrative forward.
