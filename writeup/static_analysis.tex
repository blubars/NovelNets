\section{Static-Network Analysis}

\subsection{Statistics}

\subsubsection{Degree Distribution}
Average Degree

Power law? 
$$P(k) \sim k^{-\tau}$$
$$P(k) \sim k^{-\tau}10^{-k/c}$$

TODO: 'logarithmically bin data and perform a linear regression of Log(P(5)) on log(r) to get the power law fit

\subsubsection{Assortativity}
We see an unweighted degree assortative mixing coefficient of -0.0945.
Degree assortative networks typically reflect core-periphery structures, where a dense core of highly-connected nodes is surrounded by successively less-dense periphery nodes. Degree disassortative networks, on the other hand, are more stars with high-degree nodes connected to low-degree. 
According to Newman in {\em Networks}, social networks are unusual in that they typically have a positive degree assortativity \cite{NewmanBook}. 
Therefore it is strange for us to see disassortative mixing by degree in \infinitejest, possibly indicating more of a star-like structure or fewer community structures than real-world social networks.

\subsubsection{Small World}
Comparison of diameter to real social networks?

\subsection{Modularity}

- talk about NMI
- talk about modularity

\subsubsection{Clustering}
Using the transitivity definition of clustering coefficient, we can examine the fraction of closed triads.
$$ C = \frac{\text{(number of triangles)} \cdot 3}{\text{number of connected triples}} $$

This metric disregards the edge weights, looking only at connections between characters. We find $C = 0.3930$, reflecting the relatively dense connections -- perhaps within communities such as the Tennis Academy or Halfway House -- as opposed to a tree-like structure rooted at the highest-degree (main) characters.

We can compare the clustering coefficient against the configuration model to determine if this effect is due to the degree sequence alone or perhaps reflects a conscious author choice. We find configuration model we get an average local clustering coefficient of 0$.1728$, vs $0.3930$ in the book. 

TODO: how does this compare to real social networks?

\subsection{Gender}

