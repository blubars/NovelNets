\section{Conclusion}

Our analysis demonstrates the feasibility of the naive entity co-mention method for creating networks from books. Well-known network analysis techniques are able to recover community structures and to identify important characters from the networeks our method produces. we plan to open source our entity annotation tool, as well as our dynamic network generator, to facilitate further work in this area on texts.

\section{Future Work}

There are several interesting questions to explore. We only ran our analysis with a single book, but it would be interesting to compare character interaction networks within and between different genres. For example, given a few dozen books, we could start to determine how \infinitejest's character network structure compare to other postmodern works, or how postmodern works differ from others like mysteries, historical fiction, fantasy, or short stories.

It would also be interesting to note if we can structurally identify common narrative techniques or plot devices -- for example, the climax, backstories, cliffhangers, plot twists, or stories-within-stories.

\bibliographystyle{ACM-Reference-Format}
\bibliography{references.bib}
